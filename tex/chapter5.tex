% !TeX root=../main.tex

\chapter{نتیجه‌گیری}
\label{section_conclusion}
در این پژوهش، روشی برای تولید بافت برای بازسازی سه‌بعدی از یک تصویر اشیاء با الگو‌های تکراری و دارای ساختار پیچیده، ارائه شد. این روش مبتنی بر حل یک مسئله خطی بهینه‌سازی در فضای گرادیان و در‌هم‌ریزی پواسون، برای گسترش افقی قطاع‌های الگو بود. در این پژوهش به دنبال بافت از قطاع‌های تکرار‌شونده‌ی الگو بودیم، به طوری که خطوط عمودی قابل مشاهده‌ای در مرزهای قطاع‌های مجاور وجود نداشته باشد. 
%
نشان دادیم با استفاده از این روش، اختلاف نوری یا سایر عناصر بصری را می توان در حین انتقال از یک قطاع الگو به قطاع بعدی حذف کرد، که منجر به تولید بافت نهایی جذاب‌تر از نظر بصری برای مدل سه بعدی می‌شود. نشان دادیم که با استفاده از روش ارائه‌شده، ایجاد بافت برای اشیاء با تعداد تکرار فرد در الگوی تکراری آنها، نتایج بهتری نسبت به استفاده از یک تکرار ساده از بافت جلویی در پشت دارد. مزیت دیگر رویکرد ما این بود که می‌توان آن را برای اشیاء با بخش‌های متنی در بافت خود استفاده کرد، جایی که استفاده از تکنیک آینه کردن برای یکسان‌سازی شرایط نوری باعث به هم ریختن بافت می‌شود. همانطور که در نتایج نشان داده شده است، انتخاب یک برش الگوی قابل اجرا برای نتیجه رویکرد ما بسیار مهم است. از آنجایی که روش ارائه‌شده بر نواحی مرزی قطاع متمرکز است، انتخاب یک برش الگو با تفاوت قابل‌توجه در روشنایی دو طرف قطاع منجر به ایجاد بافت‌هایی می‌شود که به مقدار مناسبی هموار نشده‌اند.

برای کارهای آینده، به جای تلاش برای برش الگو استخراج شده از بافت رویی شی، استخراج قطاع مستقیماً از تصویر شی و سپس اصلاح واپیچش این قطاع می‌تواند روشی برای دست‌یابی به نتایح مطلوب‌تری باشد. موضوع دیگری که می‌تواند برای انتخاب بهتر قطاع انجام شود، امتیاز دادن به مدل سه بعدی نهایی تولید شده بر اساس تفاوت بین مدل سه بعدی و تصویر ورودی و پیشنهاد بهترین نقاط برش در بافت جلویی استخراج شده است. یکسان‌سازی شرایط نوری در طول قطاع، به جای تمرکز فقط بر حاشیه‌های قطاع، جنبه دیگری از روش ما است که نیاز به مطالعه بیشتر دارد. اینکار می‌تواند با انتخاب ناحیه هم‌پوشانی به صورت پویا یا اضافه کردن قسمت‌های پیش‌پردازش به روند اجرای روش، انجام شود. بخش دیگری که می‌تواند باعث بهبود روش ارائه‌شده شود، پیاده‌سازی روش‌های دیگر در قسمت حل سیستم معادلات خطی نیز می‌تواند باعث افزایش سرعت این روش شده و زمینه‌ی مناسبی برای ادامه‌ی پژوهش انجام شده باشد.