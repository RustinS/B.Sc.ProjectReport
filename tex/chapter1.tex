% !TeX root=../main.tex
\chapter{مقدمه}
% دستور زیر باعث عدم‌نمایش شماره صفحه در اولین صفحهٔ این فصل می‌شود.
%\thispagestyle{empty}
\section{انگیزش}
بازسازی سه‌بعدی از دیرباز مبحث مهمی در بینایی ماشین بوده است. در این زمینه محققین به دنبال تولید مدل سه‌بعدی اشیاء مختلف، با استفاده از تصویر یا تصاویر دو‌بعدی آن شیء هستند. این مسئله به عنوان یک مسئله بنیادین در موضوعات مختلفی مانند \gls{Remote Sensing}، ناوبری، انیمیشن‌های سه‌بعدی، معماری، کمک های پزشکی و غیره، کاربرد‌های مختلفی دارد. یکی از زیرشاخه‌های سخت‌تر و پیچیده‌تر این زمینه، \gls{Single-View 3D Reconstruction} است که در آن هدف، بازسازی سه‌بعدی مدل با استفاده از تنها یک تصویر از شیء است. این زمینه امروزه در حال جذب توجه بیشتری از محققین است تا با استفاده از روش‌های نوین و سخت‌افزار‌های قوی‌تر، به نتایج بهتری در این زمینه دست یابند. این مسئله خود، شامل دو مسئله‌ی جداگانه می‌شود: مدل‌سازی \gls{Geometry}‌ی شیء و تولید \gls{Texture} مناسب با توجه به تصویر ورودی.

تحقیقات مختلفی برای مدل‌سازی هندسه‌ی سه‌بعدی شیء با استفاده از روش‌های ریاضیاتی\cite{SMH.Hosseini} و یادگیری عمیق\cite{fan2016point} صورت گرفته است که از هدف پژوهش ما خارج است. تعدادی از پژوهش‌ها نیز تلاش کرده‌اند با استفاده از یک مدل تک مرحله‌ای، هم بازسازی هندسه و هم استخراج الگو و تولید بافت را انجام دهند\cite{li2020selfsupervised}\cite{pavllo2020convolutional}. این روش‌ها در حال حاضر نتایج به نسبت مناسبی برای اشیاء با هندسه‌ی رنگی ساده دارند، اما خروجی‌های تولید‌شده برای اشیاء با الگو‌های پیچیده‌تر مطلوب نیستند. به نظر می‌رسد همچنان جداسازی مسئله‌ی بازسازی سه‌بعدی با استفاده از یک تصویر به بازسازی هندسه‌ی شیء و تولید بافت همچنان نتایج بهتری، به علت تخصصی‌تر شدن موضوع، دارد.

مسئله‌ی تولید بافت برای بازسازی سه‌بعدی از یک تصویر را می‌توان در دید جامع، به صورت کاربردی از زمینه‌ی تولید بافت از یک تصویر دانست؛ البته در بعضی از پژوهش‌های جدید تلاش شده با استفاده از هندسه‌ی سه‌بعدی مدل‌شده و تصویر دو‌بعدی شیء، عمل تولید بافت صورت گیرد\cite{oechsle2019texture}\cite{huang2020adversarial}. این روش‌ها فعلا نتیجه‌ی بهتری را ارائه نمی‌دهند و بیشتر برای بافت‌هایی که فقط از چند قسمت رنگی تشکیل شده‌اند یا الگوی‌های تکرار شونده‌ی کوچک دارند، کاربردی هستند. در زمینه‌ی تولید بافت از یک تصویر، روش‌های مختلفی توسط محققان در طول زمان ارائه شده است. به طور کلی روش‌ها معمولا به دو دسته‌ی ریاضیاتی/آماری\cite{RPN}\cite{PyramidTexSyn}\cite{FRAME} و \gls{Patch-Based}\cite{nonParam}\cite{imageQuilting}\cite{patchBasedSampling} تقسیم می‌شوند که محققین در هر دوی این دسته‌ها در حال حرکت به سوی یادگیری عمیق و \gls{Convolutional Neural Network}\cite{gatys2015texture}\cite{li2016combining} هستند. برای خلاصه‌ای از پژوهش انجام شده می‌توان به \cite{jetchev2017texture} و \cite{survey2020} رجوع کرد. این تحقیقات با اینکه نتایج مطلوبی را برای بافت‌های با الگو‌های کوچک پر‌تکرار تولید می‌کنند، در تولید بافت برای الگو‌های با \gls{Structure}‌ پیچیده و کم‌تکرار نتایج قابل قبول و مناسب ندارند. به نظر می‌رسد در کار‌های انجام‌شده توجه کمی به این نوع الگو‌ها شده است.

بخش دیگری که به نظر می‌رسد می‌تواند در تولید بافت مفید باشد اما به آن در این زمینه اهمیت کمی داده شده، بحث اتصال تصاویر در کنار یکدیگر است. کاربرد اصلی این زمینه در تولید تصاویر \gls{Panorama} است. گسترش الگو‌ها از طرفین جزو اهداف تولید بافت از یک تصویر است که در این زمینه نیز این هدف دنبال می‌شود. در این زمینه، تحقیقات مناسبی برای استفاده از \gls{Gradient Domain}\cite{Levin2004SeamlessIS}\cite{Paul2016MultiExposureAM} انجام شده است که به نظر می‌تواند در مسئله‌ی مشخص‌شده کاربرد مناسبی داشته باشد.
\section{الگوریتم پیشنهاد شده}
در این پژوهش، ما روشی برای تولید بافت از الگو‌های تکرارشونده با ساختار پیچیده و تعداد تکرار محدود ارائه می‌دهیم. در این روش تلاش‌شده با استفاده از روش‌های اتصال تصاویر سراسرنما، که در قسمت قبلی اشاره شد، قطاع‌های الگو‌ی تکرارشونده به صورت افقی گسترش یابند و بافت نهایی مناسب برای شیء حاوی این الگو‌ها تولید شود. ورودی این روش الگوی کامل قسمت رویی در تصویر اصلی شیء است که در روش ما برای استخراج قطاع تکرار شونده بریده می‌شود. برای رفع \gls{Distortion} باقیمانده در الگوی رویی، روش رفع واپیچش، به خصوص رفع \gls{Barrel Distortion}، استفاده می‌شود که بدون نیاز به مشخصات دوربین و با امکان انتخاب ضرایب، پیاده‌سازی شده است. سپس قطاع استخراج‌شده با استفاده از ویرایش در \gls{Gradient Domain}\cite{GradientShop} و در‌هم‌ریزی پوآسون\cite{PoissonImageEditing} به صورت خطی در نواحی \gls{Overlap} با قطاع همسایه خود، بهینه می‌شود تا در گذر از یک تکرار به تکرار بعدی، بافت از لحاظ رنگ و شدت نور هموار شود. این هموارسازی در دو مرحله صورت می‌گیرد تا از هموار شدن ناحیه هم‌پوشانی قطاع‌های میانی و دو طرف بافت مطمئن شویم. در انتها بافت‌نهایی با توجه به ورودی تعداد تکرار که توسط کاربر وارد شده، تولید می‌شود.

نشان داده می‌شود که این روش در تولید بافت برای اشیاء با الگوهای تکرار‌شونده‌ی دارای ساختار پیچیده و کم‌تکرار، نتایج بهتری نسبت به دیگر الگوریتم‌ها ارائه می‌دهد. به صورت دقیق‌تر، با استفاده از این روش بافت‌هایی تولید می‌شوند که می‌توانند ساختار الگوی تکرارشونده را حفظ کنند و همینطور از لحاظ ظاهری هموار و مناسب باشند. این روش به صورت خاص برای استفاده در فرآیند تولید بافت در بازسازی سه‌بعدی‌ شیء با استفاده از یک تصویر توسعه داده ‌شده و نتایج تولید‌شده از این روش در این هدف نیز آورده شده است؛ البته می‌توان از روش ارائه‌شده با هدف گسترش تصاویر تکراری به صورت افقی برای زمینه‌های دیگر نیز استفاده کرد.

مشارکت‌های اصلی ما به صورت زیر خلاصه ‌می‌شود:
\begin{itemize}
	\item 
	ما یک روش برای تولید بافت از تصویر رویی اجسام حاوی الگوی تکرارشونده‌ی با ساختار پیچیده و کم‌تکرار ارائه می‌کنیم.
	\item 
	ما با استفاده از روش‌های استفاده‌شده در تولید تصاویر سراسرنما، راه‌حلی برای مسئله‌ی تولید بافت ارائه می‌کنیم.
	\item
	ما برای رفع واپیچش تصویر، یک روش ارائه می‌دهیم که نیاز به مشخص کردن مشخصات دوربین ندارد.
	\item 
	ما برای تکرار و اتصال قطاع‌های تکرارشونده، از روش در‌هم‌ریزی پوآسون در نواحی هم‌پوشانی استفاده می‌کنیم.
	\item 
	ما روش خود را بر روی ورودی‌های مختلف اجرا می‌کنیم و نتایج به دست آمده را با دیگر روش‌ها مقایسه می‌کنیم.
	\item
	ما نتایج استفاده از  الگوریتم خود را در زمینه‌ی تولید بافت برای بازسازی سه‌بعدی از یک تصویر نمایش می‌دهیم، این نتایج را از لحاظ ویژگی‌های ظاهری بررسی کرده و در صورت نیاز با روش تکرار ساده‌ی الگوی رویی شیء برای قسمت پشتی مدل مقایسه می‌کنیم.
\end{itemize}
\section{ساختار پایان‌نامه}
در فصل دوم، تعاریف اساسی مربوط به دو زمینه‌ی تولید بافت و تولید تصاویر سراسر‌نما بیان می‌کنیم و مروری بر پیشینه‌ی تحقیق و نتایج به دست آمده از این تحقیقات خواهیم داشت. فصل سوم در برگيرنده‌ی توضیح مربوط به روش پیاده‌سازی‌شده در این پژوهش است و در این فصل توضیحات دقیق قدم‌های برداشته شده برای تولید خروجی از ورودی آورده شده است.
در فصل چهارم، ابتدا نتایج استفاده از پژوهش‌های دیگر در کنار نتایج تولید‌شده توسط روش ما، برای ورودی‌های یکسان، آورده شده است تا بتوان تحلیل و مقایسه‌ی مناسبی از عملکرد روش خود داشته باشیم. در ادامه‌ی این فصل، نتایج تولید شده از روش ارائه‌شده در بازسازی سه‌بعدی با استفاده از یک تصویر و مدل‌های سه‌بعدی تولید‌شده از بافت ‌نهایی، نمایش داده می‌شوند. برای بعضی از ورودی‌ها، نتایج تولید بافت با استفاده از تکرار ساده‌ی بافت رویی برای پشتی نیز آورده شده است تا بتوان مدل‌های سه‌بعدی تولید‌شده را مقایسه کرد.
در نهايت، در فصل پنجم، نتيجه‌گيری‌های کلی حاصل شده در اين تحقيق، انجام شده و محدوديت‌های آن مورد بحث قرار می‌گیرند. همچنین پيشنهادهایی برای ادامه‌ی مسير به علاقمندان اين حوزه‌ ارائه خواهد شد. 