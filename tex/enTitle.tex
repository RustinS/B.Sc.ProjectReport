% !TeX root=../main.tex
% در این فایل، عنوان پایان‌نامه، مشخصات خود و چکیده پایان‌نامه را به انگلیسی، وارد کنید.

%%%%%%%%%%%%%%%%%%%%%%%%%%%%%%%%%%%%
\latinuniversity{University of Tehran}
\latincollege{College of Engineering}
\latinfaculty{School of Electrical and}
\latindepartment{Computer Engineering}
\latinsubject{Software Engineering}
\latinfield{Software Engineering}
\latintitle{Texture Synthesis for 3D Shapes with Repeating Patterns Using Poisson Blending}
\firstlatinsupervisor{Prof. Hadi Moradi}
%\secondlatinsupervisor{Second Supervisor}
%\firstlatinadvisor{First Advisor}
%\secondlatinadvisor{Second Advisor}
\latinname{Rustin}
\latinsurname{Soraki}
\latinthesisdate{Jan. 2022}
\latinkeywords{Computer Vision, 3D Reconstruction, Texture Synthesis, Gradient Domain Photo Editing}
\en-abstract{
Texture synthesis and Single-View 3D reconstruction have been two attractive topics of research in Computer Vision for a long time. Both of these topics have numerous uses in various fields. An area that is gaining momentum, is the intersection of these two fields, which is texture synthesis for single-view 3d reconstruction.
%
In this task, the problem is to synthesize a full texture from a single image of the object.
%
In this study, we attempt to propose a method to produce texture for Single-View 3D reconstruction of objects with repetitive patterns on their surface. By modifying pictures in gradient space and using Poisson blending method, this approach attempted to address the differences in color and exposure on the border sections of the repeating pattern slices as a linear optimization problem.
%
The results demonstrated that for objects with repetitive patterns, it is possible to generate textures that are both aesthetically appealing and have less distortion than other commonly used approaches. 
%
The advantage of this technique over the duplication of the front-faced surface pattern for the back section of the 3D reconstructed model was demonstrated via texture synthesis for objects with an odd number of repeating surface patterns. 
%
Another advantage of this method over the mirroring technique to match lighting intensity was the ability to keep textual elements of the pattern intact.
}